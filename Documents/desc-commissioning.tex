\documentclass[modern]{desc-tex/styles/lsstdescnote}
\usepackage{desc-tex/styles/lsstdesc_macros}

%\usepackage{pdfpages}
%\usepackage{hyperref}
%\usepackage[title, titletoc, toc]{appendix}
%\usepackage{graphicx}
%\usepackage{xspace}

\newcommand{\KB}[1]{{\textcolor{orange}{(KB: #1)}}}

\begin{document}

\title{DESC Commissioning Technical Note}
\author{Members of the DESC Commissioning Task Force}
\date{\today}

\begin{abstract}

  This DESC note presents validation tests and a set of suggested fields/observation strategies that can be employed during both the ComCam and LSST Camera commissioning periods. Several on-sky observed campaigns are currently planned to evaluate the performance of the as-built system.  Commissioning is an early opportunity to begin understanding whether DESC science goals can be met with the as-built system, suggest modifications in software and/or survey strategy to optimize LSST operations, and identify areas where more algorithmic development is needed.  Many of the tests are based on requirements as specified in the DESC SRD~\cite{DESC-SRD}.  Detailed descriptions of the tests including associated code are described.  
  
 \end{abstract}

\maketitle

\noindent
\begin{center}
  \fboxsep=5pt \fbox{
    \begin{minipage}{5.25in}
      \it This document is in initial draft form.  Add your name and expand the text.
   \end{minipage}} 
 \end{center} 
\vspace{0.1in}

\section{Introduction}

Let's all enjoy commissioning the LSST.

This document details a set of validation tests and procedures proposed to be taken during the LSST commissioning period to ensure that the DESC can achieve its science goals.  As explained in Section~\ref{sec:project-tests} the tests currently being planned by the project commissioning team will assure that the LSST achieves its overall design goals. However, those requirements are not exhaustive and won't necessarily ensure that the end-to-end system will meet all DESC requirements.  By suggesting detailed tests of our own we can get early feedback on what if any modifications to our pipelines or strategy might be necessary to ensure we meet them.

Additionally, a selection of proposed observing fields staged throughout the year, would be welcome input to the project if already planned observations could be made in them which fulfilled both sets of goals.  Many of the high level DESC requirements are specified in the DESC SRD~\cite{DESC-SRD}.  In this document, we  give detailed recipes based on those requirements designed to study how well those goals are being met. We also, when possible, suggest observations fields and supply test code.

\subsection{DESC Commissioning goals and document audience}

Within the LSST project there are many requirements that must be tested formally in order to close out the construction project with the agencies and start the full survey. These requirements are specified in a set of project documents and are being built into a schedule of activities during the commissioning period. However, there is some flexibility in exactly how the on-sky observations are implemented, and the goal is to take data that will allow the LSST project and  the broader LSST community to optimize the operations of LSST from of scientific standpoint and also to enable validation tests that go beyond the normative requirements that have been written down for the project.

The DESC can give very useful input to the project by:

\begin{itemize}

\item  Specifying the highest priority validation studies related to data quality that could be undertaken with commissioning data to ensure the DESC can meet its science goals.  Here we refer not to data reduction and algorithmic improvements but rather the quality of the raw exposures themselves.

\item Supplying the project with strategies for analyzing the commissioning data to optimize operations, broadly defined (survey strategy, observatory operations, science pipelines, etc.)

\item Supplying suggested sets of suggested observation fields, as the project has flexibility in what observations are made and the time of year for given tests is not yet known.  Suggested fields should be supplied with justifications for the particular choice such as the presence of external reference catalogs, testing of a particular systematic, and so forth.
  
\end{itemize}

Then the DESC itself has a set of requirements which have been specified in the DESC SRD in order to meet our science goals with are often based on pipelines and algorithms which can be improved as the survey progresses. There is substantial overlap with the goals outlined above.  In the case of DESC specific needs commissioning goals include:

\begin{itemize}

\item Placing quantitative requirements for DESC science on quantities such as galaxy shapes, galaxy detection completeness, galaxy photometry, deblending, photometric redshifts, star-galaxy separation, photometry for transients, accuracy of the survey mask geometry representation, etc.  These are the quantities which in many cases have already been specified in the DESC SRD, or at least have been identified as needing a quantitative metric.  In most cases these requirements are statements about the performance of algorithms and pipe-lines and can be expected to improve in time.

\item Designing and supplying the specific tests (including code) which can evaluate these metrics.

\item Establishing a structure from which DESC members can contribute to validation studies specifically targeted at dark energy science, thus making a clear case for why work during the commissioning period should be supported by the agencies for those members.
  
\end{itemize}

The work on this DESC technical note will be done in the open and work will be visible to the other LSST science collaborations.  When complete, we will make our recommendations and tests available to the LSST project for inclusion in the commissioning effort.

\section{Summary of Planned Project Tests}
\label{sec:project-tests}

{\bf \it Note: If you would like to read (for example) the project
  document LSE-79, you can use the URL: https://ls.st/LSE-79.}

The LSST Project has a well defined commissioning plan~\cite{LSE-79} and is planning on verifying that the instrument is operating as designed.  These requirements are specified in a set of project documents ( e.g., the Observatory System Specification (LSE-30)~\cite{LSE-30}, the LSST System Requirements (LSE-29)~\cite{LSE-29}, and the LSST Systems Science Requirements Document (LPM-17)~\cite{LPM-17}. The project understands that the formal verification requirements are not exhaustive and welcomes input on validation studies, i.e., test specifications that go beyond the normative requirements. The Project also welcomes input on the specific observations that could be done during commissioning. It can, however, not guarantee that all suggestions will be implemented.

{\it [Should we list current planned tests here so as to make sure we don't have overlaps and give people a sense of what tests might also be able to use suggested fields?  OTOH this could be quite a long list?  Some template language if we want to do this below:] }

Here we list currently planned tests to be undertaken by the project science validation team along with links to more detailed information.  The purpose of including this list is two-fold: first, to avoid duplication in newly suggested tests and second to understand which new suggested observation fields might usable for these tests which are already planned.

\noindent
{\bf List of currently planned project requirement tests}
\begin{enumerate}
\item Prove that pie is better than cake.
\item Eat more pie.
\item I don't care what you say, pie is better than cake.
\end{enumerate}

\section{DESC Commissioning Tests}

If you were writing a DESC paper, what plot would you put in into your paper to show that you could do the science you are going to report? What null tests would you demonstrate?  What performance metrics would you highlight?  The goal of the commissioning tests are to systematically undertake these tests early.

As a start, we would like each group to gather requirements necessary to do their science.  Many of the requirements we need are already specified or listed in the DESC SRD; and, to help you start, in each section below some requirements taken from the DESC SRD are listed for each probe.  You can use these DESC SRD requirements in each section as a starting point, but you might want to remove them if you think the tests are not accessible during the commissioning period. You will likely want to add requirements based on the suggestions in the SRD appendices.   Those appendices list many requirements for which we should have specific requirements but have not yet calculated numbers.   Additionally, your working group will likely come up with new requirements based on (fro example) pixel level requirements that may not be specified in the SRD.

{\it Instructions: For each group start by writing an introduction for your section of what you would like to test.  Then, begin to write down requirements that you think are necessary for Dark Energy Science that will not be addressed in the normative project commissioning plan.  Are there other important additional requirements you think we should address not in ether the DESC SRD or associated project documents?  If so, let's try to write down those tests also.

As a next step: consider what samples would be necessary to characterize the instrument and software performance and then, based on some of the guidelines in Appendix A., try to outline tests that would correspond to each requirement you wrote down. }

\subsection{Large Scale Structure}

\begin{itemize}
\item Systematic uncertainty in the mean redshift of each tomographic bin shall not exceed $0.003 (1 + z)$ in the Y10 DESC LSS analysis.
\item Systematic uncertainty in the photometric redshift scatter $\sigma_z$ shall not exceed $0.03 (1 + z)$ in the Y10 DESC LSS analysis.
\end{itemize}

\subsection{Weak Lensing (3x2pt)}

\begin{itemize}
\item Systematic uncertainty in the mean redshift of each source tomographic bin shall not exceed $0.001 (1 + z)$ in the Y10 DESC WL analysis.
\item Systematic uncertainty in the source photometric redshift scatter $\sigma_z$  shall not exceed $0.003 (1 + z) $ in the Y10 DESC WL analysis.
\item Systematic uncertainty in the redshift-dependent shear calibration shall not exceed 0.003 in the Y10 DESC WL analysis.
\item Systematic uncertainty in the PSF model size defined using the trace of the second moment matrix shall not exceed 0.1\% in the Y10 DESC WL analysis.
\item Systematic uncertainty in the stellar contamination of the source sample shall not exceed 0.1\% in the Y10 DESC WL analysis.
\end{itemize}

\subsection{Galaxy clusters}

\begin{itemize}
\item Systematic uncertainty in the mean redshift of each source tomographic bin shall not exceed $ 0.001( 1 + z)$ in the Y10 DESC CL analysis.
\item Systematic uncertainty in the source photometric redshift scatter shall not exceed $0.005 (1 + z)$ in the Y10 DESC CL analysis.
\item Systematic uncertainty in the redshift-dependent shear calibration shall not exceed 0.008 in the Y10 DESC CL analysis.
\end{itemize}

\subsection{Supernovae}


\begin{itemize}
\item Systematic uncertainty in the griz-filter zero points shall not exceed 1 mmag in the Y10 DESC SN analysis. As the griz requirements represent an ambitious improvement versus the LSST SRD (5 mmag in griz), an alternative way to meet this requirement is to improve our analysis methods for all probes until the LSST SRD requirement is sufficient.
\item Systematic uncertainty in the griz-filter mean wavelength shall not exceed 1\AA{} in the Y10 DESC SN analysis.
\item  Systematic uncertainty in the wavelength-dependent flux calibration shall not exceed a slope of 4.4 mmag per 5500\AA{} in wavelength in the Y10 DESC SN analysis.
\item Systematic uncertainty in the light curve modeling should not exceed 22\% of current SALT2 model errors in the Y1 DESC SN analysis.
\item Detailed requirement SN5 (Y10): Systematic uncertainty in Milky Way extinction corrections shall not exceed 30\% of current systematic Galactic extinction uncertainties in the Y10 DESC SN analysis.
\end{itemize}


\subsection{Strong Lensing}

The DESC SRD does not currently specify any requirements on Strong Lensing.

\subsection{Combined Probes and other requirements}

The DESC SRD does currently define any relevant requirements for commissioning.

\vspace{1.0in}

There may be other tests that are not currently covered by the science cases above that we should specify.  They are enumerated below.

{\it [Please fill in as needed.]}

\subsection{End-to-End and other tests}
\subsection{Photometric Calibration}
\subsection{Sensor related tests}

\subsection{Observing Strategy related tests}

There may be recommendations from the DESC OSTF that would be good to actually test during the commissioning period such as strategies for footprint or exposure time.

{\it We will wait until after the observing strategy white paper has been turned in to engage with this work}


\appendix
\section{Test Templates}

What makes a good test?  Here we list the properties necessary to complete specify a test and to allow others to carry it out if necessary.

{\bf Characteristics of tests }
\begin{enumerate}
\item Clearly defined set of observations (including fields, filters, observing strategy).
\item Image measurements that will be undertaken that are different than those in the DPDD (if any).
\item Additional data sets required as input to the test or metric.
\item Metric/test that would be applied to the data.
\item Description of Value or plot  that we would like to produce.
\item Sample notebooks for the test with simulation or pre-cursor data.
\end{enumerate}

{\it [We should make a template that everyone can follow so that we have a consistent set of tests.]}

\begin{acknowledgments}
The DESC acknowledges ongoing support from the Institut National de Physique Nucl\'eaire et de Physique des Particules in France; the Science \& Technology Facilities Council in the United Kingdom; and the Department of Energy, the National Science Foundation, and the LSST Corporation in the United States.  DESC uses resources of the IN2P3 Computing Center (CC-IN2P3--Lyon/Villeurbanne - France) funded by the Centre National de la Recherche Scientifique; the National Energy Research Scientific Computing Center, a DOE Office of Science User Facility supported by the Office of Science of the U.S.\ Department of Energy under Contract No.\ DE-AC02-05CH11231; STFC DiRAC HPC Facilities, funded by UK BIS National E-infrastructure capital grants; and the UK particle physics grid, supported by the GridPP Collaboration.  This work was performed in part under DOE Contract DE-AC02-76SF00515.

\end{acknowledgments}

\bibliography{desc-tex/bib/lsstdesc}

\end{document}