\documentclass[modern]{desc-tex/styles/lsstdescnote}
\usepackage{desc-tex/styles/lsstdesc_macros}

%\usepackage{pdfpages}
%\usepackage{hyperref}
%\usepackage[title, titletoc, toc]{appendix}
%\usepackage{graphicx}
%\usepackage{xspace}

\begin{document}

\title{DESC Commissioning Technical Note}
\author{Members of the DESC Commissioning Task Force}
\date{\today}


\begin{abstract}
  In this DESC note we suggest survey fields and tests that the
  commissioning team can employ during both the ComCam and LSST Camera
  commissioning periods.  These fields and tests will cover a range ofhttps://v2.overleaf.com/4187239969gbpmxhxrpjnp
  times of year and will supply important test data to validate that
  all of our science goals are being met.  They will be tied to the
  LSST system requirements as enumerated in the project JIRA system.
\end{abstract}

\maketitle

\noindent
\begin{center}
  \fboxsep=5pt \fbox{\begin{minipage}{5.25in} \it This document is in
      initial draft form.  Add your name and expand the text.
   \end{minipage}} 
 \end{center} 
\vspace{0.1in}

\section{Introduction}

Let's all enjoy building the LSST.

\section{Summary of project tests}

What LSST Project is planning on verifying (the normative requirements, e.g., OSS, LSR, Project SRD).
Project knows that the formal verification requirements are not exhaustive.
Welcome input on validation studies, i.e., test specifications that go beyond the normative requirements.
Project would welcome input on the observations that would be done during commissioning (cannot gaurantee that Project would do.)

\begin{enumerate}
\item Test A
\item Test B
\end{enumerate}

what other tests do we need. What is the plot you would put in your paper to show that you could do the science you are going to report in the paper. What are the null tests etc.


\begin{acknowledgments}
\input{desc-tex/ack/standard}
\end{acknowledgments}
  
\bibliography{desc-tex/bib/lsstdesc}


\appendix

Value added tests that can be done with the data

What makes a good test
- clearly defined set of observations (including fields, filters, observing strategy)
- image measurements that will be undertaken that are different than those in the DPDD (if any)
- additional data sets required as input to the test or metric
- metric/test that would be applied to the data
- value (or plot) that we would need to achieve
- notebooks for the test

\end{document}