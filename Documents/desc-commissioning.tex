\documentclass[modern]{desc-tex/styles/lsstdescnote}
\usepackage{desc-tex/styles/lsstdesc_macros}

%\usepackage{pdfpages}
%\usepackage{hyperref}
%\usepackage[title, titletoc, toc]{appendix}
%\usepackage{graphicx}
%\usepackage{xspace}

\begin{document}

\title{DESC Commissioning Technical Note}
\author{Members of the DESC Commissioning Task Force}
\date{\today}

\begin{abstract}
  In this DESC note we suggest survey fields and tests that the
  commissioning team can employ during both the ComCam and LSST Camera
  commissioning periods. These fields and tests will cover a range of
  times of year and will supply important test data to validate that
  all DESC science goals are being met. Many of the tests are
  implementation descriptions of requirements as specified in the DESC
  SRC~\cite{DESC-SRD}. \end{abstract}
\maketitle

\noindent
\begin{center}
  \fboxsep=5pt \fbox{
    \begin{minipage}{5.25in}
      \it This document is in initial draft form.  Add your name and expand the text.
   \end{minipage}} 
 \end{center} 
\vspace{0.1in}

\section{Introduction}

Let's all enjoy commissioning the LSST.

\section{Summary of project tests}

What LSST Project is planning on verifying (the normative
requirements, e.g., OSS, LSR, Project SRD). The project knows that the
formal verification requirements are not exhaustive and welcomes input
on validation studies, i.e., test specifications that go beyond the
normative requirements. The project would also welcome input on the
observations that would be done during commissioning although it
cannot guarantee that all suggestions will be undertaken.


\begin{enumerate}
\item Test A
\item Test B
\end{enumerate}

\section{DESC Commissioning Tests}

What other tests do we need. What is the plot you would put in your paper to show that you could do the science you are going to report in the paper. What are the null tests etc.

\subsection{Large Scale Structure}
\subsection{Weak Lensing (3x2pt)}
\subsection{Galaxy clusters}
\subsection{Supernovae}
\subsection{Strong Lensing}
\subsection{Combined Probes and other requirements}

Technical tests that are not covered by the science cases above.

\subsection{End-to-End and other tests}
\subsection{Photo-metric Calibration}
\subsection{Sensor related tests}

\appendix

Value added tests that can be done with the data

What makes a good test?

\begin{enumerate}
\item Clearly defined set of observations (including fields, filters, observing strategy)
\item Image measurements that will be undertaken that are different than those in the DPDD (if any)
\item Additional data sets required as input to the test or metric
\item Metric/test that would be applied to the data
\item Value (or plot) that we would need to achieve
\item Notebooks for the test
\end{enumerate}

\begin{acknowledgments}
% \input{desc-tex/ack/standard}
\end{acknowledgments}
  
\bibliography{desc-tex/bib/lsstdesc}

\end{document}